\documentclass[10pt,a4paper]{article}
\usepackage{geometry}
\geometry{top=1in}
\usepackage{url}

\title{Total order and partial order planning, and GRAPHPLAN}
\author{Omar A. Serrano}

\begin{document}

\maketitle

Research in AI planning and search has led to the development of many techniques
and algorithms. Three of these techniques are total-order planning,
partial-order planning, and GRAPHPLAN.

The earliest planners used total-order planning, also known as linear
programming \cite{1}. The technique specifies the complete sequence of actions
for the task at once \cite{2}, making it an incomplete approach, because it
cannot distinguish between two competing plans, but complete planners must be
able to combine squences of actions from different subplans \cite{1}.

Partial-order planning emerged as a solution to the problem of mixing actions
from different subplans, with key traits of being able to detect conflics and
preventing interference between conditions that have been met \cite{1}. A
partial-order does not specify the order of actions when the order is not
important, but does determine all the actions for a plan, and it consits of four
components: a set of actions, a partial order of the set of actions, a set of
causal relations between the set of actions and preconditions for other actions,
and a set of open preconditions that are not satisfied by the current plan of
actions \cite{2}. The disadvantage of partial-order planning is that the
approach tends to be computationally more expensive because the algorithm is
more complex, and thus may not be well suited in circumstances were the planning
agent (i.e., a robot) needs to minimize power consumption \cite{2}.

GRAPHPLAN, developed in the mid 1990s, departs from state space graph search by
representing the space as nodes of actions and facts arranged in alternate
levels \cite{3}. The first level contains the facts representing the intial
state, the next level is built of actions that have the intial state as a
precondition, and then the next level is built of the states or facts that are
consequent of the actions in the previous level \cite{3}. The algorithm
continues in this way, iteratively building up one level of before building the
next as it proves there are no solutions in the current level.

Even though the technique used by GRAPHPLAN is a more modern technique than
partial-order planning, and partial-order planning is more modern than
total-order planning, no technique is wholly superior than the others.
Total-order planning, for example, which lacks completeness when planning
requires interleaving squence of actions, is complete for problems characterized
by serializable subgoas \cite{1}, and thus for this kind of problems it may be
more efficient than a partial-ordering solution, which tend to be more
computationally expensive. More generally, techniques that use search do better
with problems that have solutions that can be found without backtracking,
whereas problems that require backtracking are better suited to constraint-based
solutions, of which GRAPHPLAN is an example \cite{1}.

\begin{thebibliography}{9}
\bibitem{1}
    Russell, Stuart, Norvig, Peter. 
    \emph{Artificial Intelligence: A Modern Approach}.
    3rd Ed. New Jersey: Pearson Education, 2010. Print.
\bibitem{2}
    \url{https://en.wikipedia.org/wiki/Partial-order_planning}
\bibitem{3}
    \url{https://en.wikipedia.org/wiki/Graphplan}
\end{thebibliography}

\end{document}
