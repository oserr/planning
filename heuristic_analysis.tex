\documentclass[10pt,a4paper]{article}
\usepackage{listings}
\usepackage{xcolor}
\usepackage{mdframed}
\usepackage{geometry}
\geometry{top=1in}
\usepackage{multirow}

\pagestyle{headings}
\definecolor{light-grey}{gray}{0.95}

\mdfsetup{frametitlealignment=\center}

\title{Analyis for planning project}
\author{Omar A. Serrano}

\begin{document}

\maketitle

\section*{Analysis of uninformed search}

In all three problems, DFS expands less nodes than BFS, which expands less nodes
than UCS. BFS and UCS produce plans with equivalent lengths, which are
significantly cheaper than the plans produced by DFS. DFS produces plans quicker
than BFS, which produces plans quicker than UCS. The relationship between the
number of nodes BFS and DFS expand seems to correlate with the total time it
takes for each search algorithm to produce a plan. With problem 1, for example,
BFS takes a little more than twice what DFS takes, which corresponds to the
number of nodes they expand. The same, however, does not apply to BFS and UCS.
Even though UCS expands about 50\% more nodes than BFS, it takes more than three
times to compute a plan, with the exception of problem 1.

Even though we cannot expect for BFS to always expand more nodes than DFS, it
will happen if the goal state is toward the left and bottom of a tree, because
those are the paths that DFS explores first. For example, if the height of the
tree is $h_{tree}$ and the goal state is the left most leaf in a complete
tree, then DFS will only expand $h_{tree}$ nodes, while BFS would expand
every node except the nodes in last level \cite{1}.

\hfill \break

\begin{tabular}{|c|l|l|l|r|}
    \hline
    & \textbf{Search} & \textbf{Expansions}
    & \textbf{Plan Length} & \textbf{Time (sec)} \\
    \hline
    \multirow{4}{*}{\textit{Problem 1}} & BFS & 43 & 6 & 0.02587 \\
    \cline{2-5}
    & DFS & 21 & 20 & 0.01181 \\
    \cline{2-5}
    & UCS & 55 & 6 & 0.03291 \\
    \cline{2-5}
    & A* \textit{IgnorePreconditions} & 41 & 6 & 0.02637 \\
    \cline{2-5}
    & A* \textit{LevelSum} & 11 & 6 & 3.67303 \\
    \hline
    \multirow{4}{*}{\textit{Problem 2}} & BFS & 3343 & 9 & 12.07154 \\
    \cline{2-5}
    & DFS & 624 & 619 & 3.00142 \\
    \cline{2-5}
    & UCS & 4853 & 9 & 42.40486 \\
    \cline{2-5}
    & A* \textit{IgnorePreconditions} & 1506 & 9 & 10.82272 \\
    \cline{2-5}
    & A* \textit{LevelSum} & 86 & 9 & 1357.63188 \\
    \hline
    \multirow{4}{*}{\textit{Problem 3}} & BFS & 14663 & 12 & 90.45000 \\
    \cline{2-5}
    & DFS & 408 & 392 & 1.54737 \\
    \cline{2-5}
    & UCS & 18223 & 12 & 335.00286 \\
    \cline{2-5}
    & A* \textit{IgnorePreconditions} & 5118 & 12 & 73.72976 \\
    \cline{2-5}
    & A* \textit{LevelSum} & 414 & 12 & 10100.00274 \\
    \hline
\end{tabular}

\section*{Analysis of informed search}

In all three problems, \textit{IgnorePreconditions} expanded more nodes than
\textit{LevelSum}, but it expands significantly more nodes in problems 2 and 3,
by more than 1000\%. Both produce plans of equal length, but \textit{LevelSum}
takes more time to compute the plan, especially in problems 1 and 2, where
it takes more than 1000\% of the time taken by \textit{IgnorePreconditons}.

\begin{mdframed}[frametitle={Problem 1: Optimal Plan},
                 backgroundcolor=light-grey, roundcorner=10pt,
                 leftmargin=1, rightmargin=1,
                 innerleftmargin=15, innerrightmargin=15,
                 innertopmargin=15, innerbottommargin=15,
                 outerlinewidth=1, linecolor=light-grey]
\begin{lstlisting}
Load(C1, P1, SFO)
Load(C2, P2, JFK)
Fly(P2, JFK, SFO)
Unload(C2, P2, SFO)
Fly(P1, SFO, JFK)
Unload(C1, P1, JFK)
\end{lstlisting}
\end{mdframed}

\section*{Informed vs. uninformed search}

Ultimately, there is no search algorithm that performs better than all the
others in every aspect. \textit{LevelSum}, for example, performs very well,
on average, in terms of the total number of nodes expanded, and the optimality
of the plan, however, the time it takes to produce a plan can increase
dramatically, taking substantially more than the other algorithms on problems
2 and 3. DFS, on the othe hand, produces the least optimal plans, but it is the
quickest of the algorithms to produce a plan, and in two out of three problems
it expands less nodes than everyone else.

\begin{mdframed}[frametitle={Problem 2: Optimal Plan},
                 backgroundcolor=light-grey, roundcorner=10pt,
                 leftmargin=1, rightmargin=1,
                 innerleftmargin=15, innerrightmargin=15,
                 innertopmargin=15, innerbottommargin=15,
                 outerlinewidth=1, linecolor=light-grey]
\begin{lstlisting}
Load(C1, P1, SFO)
Load(C2, P2, JFK)
Load(C3, P3, ATL)
Fly(P2, JFK, SFO)
Unload(C2, P2, SFO)
Fly(P1, SFO, JFK)
Unload(C1, P1, JFK)
Fly(P3, ATL, SFO)
Unload(C3, P3, SFO)
\end{lstlisting}
\end{mdframed}


\begin{mdframed}[frametitle={Problem 3: Optimal Plan},
                 backgroundcolor=light-grey, roundcorner=10pt,
                 leftmargin=1, rightmargin=1,
                 innerleftmargin=15, innerrightmargin=15,
                 innertopmargin=15, innerbottommargin=15,
                 outerlinewidth=1, linecolor=light-grey]
\begin{lstlisting}
Load(C1, P1, SFO)
Load(C2, P2, JFK)
Fly(P2, JFK, ORD)
Load(C4, P2, ORD)
Fly(P1, SFO, ATL)
Load(C3, P1, ATL)
Fly(P1, ATL, JFK)
Unload(C1, P1, JFK)
Unload(C3, P1, JFK)
Fly(P2, ORD, SFO)
Unload(C2, P2, SFO)
Unload(C4, P2, SFO)
\end{lstlisting}
\end{mdframed}

\begin{thebibliography}{9}
\bibitem{1}
    Russell, Stuart, Norvig, Peter.
    \emph{Artificial Intelligence: A Modern Approach}.
    3rd Ed. New Jersey: Pearson Education, 2010. Print.
\end{thebibliography}

\end{document}
